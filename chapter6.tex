\chapter{Conclusion}
\label{chap6:title}

The aim of the research was to analyse the communication model between SCSI controller and disks and find out the optimal way of sending WRITE and WRITE SAME commands. Several strategies of sending commands were applied, which gave the food for understanding where the bottlenecks are hidden. Moreover, these results supported to build the dynamic load balancing system, which can find the optimal values for commands.

The tests showed that if we send commands in series, the erasure time increases linearly. After finish of sending consecutive commands we decided to apply the parallel strategy. SCSI controller and disks handled it without any problems that is why we realized all the later experiments with the parallel strategy. The results gave a possibility to make several conclusions. First of all, usage of WRITE SAME command allows to erase one 18.2 GB disk by sending only 304 kB of information and the time of the erasure depends directly from the disk speed. Secondly, the bottleneck of sending WRITE commands is the bus, because with that command we need to send the amount of data, which exceeds the capacity of the disks. However, this fact gave a possibility to vary the parameters and different tests helped to find out that the size of sending buffer played the main role. To make the process faster the buffer size should fit in the allocation memory of the SCSI controller that was found in the Linux driver. These facts played very important roles in building the suggested dynamic load balancing system, because it helps to calculate the optimal values for sending the commands for the erasure.

For the future research we suggest to figure out if we discussed all the parameters of this communication. For example, there is the question why the erasure times of the disks with the same properties are so different during single erasure. Moreover, it would be a great step forward to balance the results of single erasure, which gives a possibility to make lower the value of complete erasure. In this research only one SCSI controller was considered and it is possible that another controller will have similar behaviour, but with its own specificities.