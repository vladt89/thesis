\chapter{Introduction}
\label{chap1:title}
\nomenclature{IT}{Information Technology}
\nomenclature{SCSI}{Small Computer System Interface}

Information technology (IT) is growing up so fast nowadays that we cannot even compare what was 10 years ago and what is now. Current situation shows that every month, every day, every minute IT-community goes forward and the steps are so huge that some products, which were popular couple years ago, are already outdated and not supported. All over the world people start using more mobile phones, laptops, electronic books and so on. This is the reason why the amount of information is increasing so fast. User expects that the device will work without any errors and delays. Moreover, user wants that it will be easy to use and the information will keep privacy that nobody can get the access to it. From technical side the systems start to be very complicated, because the processes start to take more memory and the commands start to mix up, that is why for the device it started to be more difficult to handle all these things. 

Every computation system has the limits of speed, memory or some other parameters. However, if the program even works correct the speed can be decreased by some reason that can be not because of software. It is also possible to happen that the application would start taking too much memory. That is why IT-community started to focus on these problems a lot. In this research we will consider the optimization, which is a general technique of finding the solutions for this topic. In mathematics and computer science, optimization is the selection of a best element from some set of available alternatives with regard to some criteria. In current work we will try to find the optimal speed for the erasure process, which depends on a lot of parameters. Load balancing methods, which is part of optimization theory, will help us to find the right solution.

The methodology called \emph{Load balancing} was invented not so long time ago and serves for making the computation system faster with less usage of the resources \cite{load_bal}. Mostly this technology is using for the Internet services \cite{dyn_bal_web}, for example, one of the most used common applications of load balancing is to provide a single Internet service from multiple servers. In this paper we will consider load balancing methodology on the example of communication between Small Computer System Interface (SCSI) controller and different amount of SCSI disks. 

Lets imagine that some program sends the SCSI WRITE and SCSI WRITE SAME commands from the computer though the controller to the disks to perform the complete erasure. There are a lot of parameters, which can prevent the communication speed between the computer program and the disks. Moreover, some more parameters appear for consideration because we have a SCSI controller in the middle. In the thesis we will apply several load balancing strategies and test the communication model with different parameters. The task is to figure out the fastest technique of erasure.

The thesis discusses the parameters, which are more important for the speed of the erasure. Tests with HP Smart Array 642 controller show how the speed depends on such parameters as transfer length, disk cache, disk capacity, amount of the disks and others. Moreover, the thesis suggests to apply special dynamic load balancing system to realize the erasure faster. The aim of this system is to find the optimal transfer length for the buffer, which allows to send the data in optimal way. 

Current thesis has 6 chapters, including concurrent parts as introduction and conclusion. The second chapter gives an open view to the basic concepts of the thesis. Chapter 3 discusses the mathematical way of the problem and considers all parameters, which can influence on the communication process between the devices. The fourth chapter is one of the most important ones, because we describe there the main idea of application the dynamic load balancing system. The principal concepts of new system are based on the chapter 5, where we present different results from testing. Chapter 6 concludes all work and sum up the ideas from the thesis.

