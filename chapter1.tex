\chapter{Introduction}
\label{chap1:title}
\nomenclature{IT}{Information Technology}
\nomenclature{SCSI}{Small Computer System Interface}

Information technology (IT) is growing up so fast nowadays that we can not even compare what was 10 years ago and what is now. Current situation shows that every month, every day, every minute IT-community goes forward and the steps are so huge that the product, which was so popular 1 or 2 years ago is already old and people even could not use it, because the company already stopped supporting this product. All over the world people start using more and more mobile phones, laptops, read books and so on. This is the reason why the amount of information is increasing so fast. From the users side the devices should work without any errors and delays. Moreover, user wants that it will be easy to use and the information will keep privacy that nobody can get the access to it. From technical side the systems start to be very complicated, because the processes start to take more memory and the commands start to mix up, that is why for the device it started to be more difficult to handle all these things. Every computation system has the limits of speeding, memory or some other parameters, and even if the program works correct it can be decreased by some reason or can start taking too much memory. That is why IT-community started to focus on that problem a lot. 

The methodology called \emph{Load balancing} was invented not so long time ago and serves for making the computation system faster with less usage of the resources \cite{load_bal}. Mostly this technology is using for the Internet services \cite{dyn_bal_web}, for example, one of the most used common applications of load balancing is to provide a single Internet service from multiple servers. In this paper load balancing methodology will be considered on the example of communication between Small Computer System Interface (SCSI) controller and different amount of SCSI disks. From the computer the program sends the SCSI WRITE and SCSI WRITE SAME commands to the disks to make complete erasure. The problem is to figure out which technique is faster.

The thesis discusses which parameters are more important for the speed of the erasure. Tests with HP Smart Array 642 controller shows how the speed depends on such parameters as transfer length, disk cache, disk capacity, amount of the disks and so on. Moreover, the thesis suggests to apply special dynamic load balancing system to make the erasure faster. The aim of this system is finding the optimal transfer length for the buffer, which allows to send the data in optimal way. 

