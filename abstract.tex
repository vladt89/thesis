{\bf Abstract}\\



Load balancing is computer networking method, which makes systems faster and in the same time use less resources. Almost every system can be improved by less resource usage or acceleration of speed, that is why this methodology is an important part of any computation system. The thesis describes the model of communication between SCSI controller and SCSI disks by different write commands. The application system sends SCSI WRITE and WRITE SAME commands to the disks to make the complete erasure of the disks. The aim of the research is to figure out if the parallel strategy of overwriting the disks is optimal. The thesis describes what parameters can prevent the fast speed of erasure and we can avoid these parameters if it is possible. The model, presenting in the thesis, is subjective, which means that it can be different for other systems, but works properly in this case. The results of the thesis show that disk speed is the limiting factor for the disks, which are erased by WRITE SAME command. Using the WRITE command the bus is the bottleneck, but by varying transfer length of the buffer it is possible to find the optimal way of sending the commands depending on the amount of disks.





{\bf Keywords:}\\
Load balancing, RAID, SCSI, pass through commands.
